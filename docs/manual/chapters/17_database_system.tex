\chapter{Asteroid Database System}
\label{ch:database_system}

This chapter describes the comprehensive asteroid database infrastructure implemented in \ioccultcalc{} for efficient large-scale occultation searches.

\section{Overview}

The database system enables:
\begin{itemize}
    \item Storage of 1.3+ million asteroid orbits
    \item Fast filtering by orbital parameters
    \item Efficient query by designation or number
    \item Integration with AstDyS and MPC sources
\end{itemize}

\section{Data Sources}

\subsection{AstDyS (Asteroids Dynamic Site)}

Primary source for high-quality orbits:
\begin{itemize}
    \item URL: \texttt{https://newton.spacedys.com/astdys/}
    \item Updated daily
    \item Provides: orbital elements, covariance matrices, observations
    \item Format: Proprietary text format
\end{itemize}

\subsection{MPC (Minor Planet Center)}

Comprehensive catalog:
\begin{itemize}
    \item URL: \texttt{https://minorplanetcenter.net/}
    \item MPCORB: Complete orbital element database
    \item $>1.3$ million asteroids
    \item Format: Fixed-width text
\end{itemize}

\section{Database Schema}

SQLite database with optimized schema:

\begin{lstlisting}[language=SQL]
CREATE TABLE asteroids (
    id INTEGER PRIMARY KEY,
    designation TEXT UNIQUE,
    number INTEGER,
    epoch REAL,
    a REAL,           -- Semi-major axis (AU)
    e REAL,           -- Eccentricity
    i REAL,           -- Inclination (deg)
    omega REAL,       -- Arg. perihelion (deg)
    Omega REAL,       -- Long. asc. node (deg)
    M REAL,           -- Mean anomaly (deg)
    H REAL,           -- Absolute magnitude
    G REAL,           -- Slope parameter
    source TEXT
);

CREATE INDEX idx_designation ON asteroids(designation);
CREATE INDEX idx_number ON asteroids(number);
CREATE INDEX idx_semimajor ON asteroids(a);
CREATE INDEX idx_magnitude ON asteroids(H);
\end{lstlisting}

\section{Filtering System}

\subsection{Orbital Element Filters}

\begin{lstlisting}[language=C++]
struct AsteroidFilter {
    std::pair<double, double> a_range;      // AU
    std::pair<double, double> e_range;      // dimensionless
    std::pair<double, double> i_range;      // degrees
    std::pair<double, double> H_range;      // magnitude
    std::vector<std::string> groups;        // NEA, MBA, etc.
};
\end{lstlisting}

\subsection{Predefined Groups}

\begin{itemize}
    \item \textbf{NEA}: Near-Earth Asteroids ($a < 1.3$ AU)
    \item \textbf{MBA}: Main Belt ($2.0 < a < 3.3$ AU)
    \item \textbf{Trojans}: Jupiter Trojans ($a \approx 5.2$ AU)
    \item \textbf{TNO}: Trans-Neptunian Objects ($a > 30$ AU)
\end{itemize}

\section{Performance Optimization}

\subsection{Indexing Strategy}

Multiple indexes for common queries:
\begin{itemize}
    \item Designation: B-tree index (fast exact match)
    \item Semi-major axis: Range queries for orbital groups
    \item Magnitude: Brightness filtering
\end{itemize}

\subsection{Query Performance}

Typical performance (1.3M asteroid database):
\begin{itemize}
    \item By designation: $<1$ ms
    \item By number: $<1$ ms
    \item By orbital range: $<50$ ms
    \item Full scan: $\sim 500$ ms
\end{itemize}

\section{Implementation}

See \texttt{src/asteroid\_database.cpp} for complete implementation.

Key classes:
\begin{itemize}
    \item \texttt{AsteroidDatabase}: Main interface
    \item \texttt{AsteroidFilter}: Filtering logic
    \item \texttt{DataManager}: Download and update
\end{itemize}
